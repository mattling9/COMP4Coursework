\chapter{Analysis}

\section{Introduction}

\subsection{Client Identification}

	\begin{flushleft}

My client is Tom Henderson; he is thirty seven years old and is the principle veterinary surgeon and managing director of Beacon Veterinary. Tom graduated from Melbourne University, Victoria, Australia in 2000 and found employment with the Beacon Vet Centre in Aspatria in 2003. Since then Tom has brought partnership in the practice. \par


	\end{flushleft}

\subsection{Define the current system}

	\begin{flushleft}
Despite being a veterinary center, Beacon vets sold roughly a dozen products, such as Vetericyn, a non toxic cut / wound treatment for animals. Because only a dozen items were being sold and only 1-3 products were sold at once, keeping track of the stock of each item was an extremely easy process./par

Last year Beacon Vets expanded on the range of the products they sold, providing a wide variety of foods and treatments for all sorts of animals such as dogs, cats and rabbits etc. Although Beacon Vets, greatly expanded on the products they sold, they did not change their system of keeping track of stock. The current system is a Microsoft Excel spreadsheet containing the information of the product, the selling price, the current amount in the front of store, the current amount being stored in the back of the store. This current system was useful and easy to use when handling a dozen products but has become difficult and confusing now many more products are being sold. here is the current Edexel Spreadsheet in use:\par

Every time products are sold, the document is manually changed.For example if the veternary sold 1 bag of dietry dog food, the Front of house stock would go from 7 to 6. When the stock in the front of house gets low, products are moved from the back of house to the front of house for the customers to buy. when the total stock gets low, an order of that specific product is made.\par
	\end{flushleft}

\subsection{Describe the problems}

	\begin{flushleft}
There are many problems with the current system. Having to keep manually changing the document every time a product is sold, moved to the front of house or if a delivery comes in. Editing the document becomes tedious and sometimes errors are made changing the values. Selling the product and changing the document are two different processes. This is a problem as both these processes can be combined into one to improve the system. \par

The current operating system on the computer is Windows 7. The program Microsoft Excel which may not be compatable with other operating systems such as Mac or Linux. This is a problem with the current system because if a new Computer is brought with a different operating system, the current system may not be compatable and usable on the new computer. If the current system is used for a long period of time (for example 5 years), The version of Excel the system was created in may not be compatible with an up to date version of Excel. This could also be problematic because it would mean the current system would stop functioning. A similar problem would be that the current system was created using an upto date version of Excel and the current version of Excel on the computer is not upto date, which could result in the System not being compatible with the version of Excel installed on the computer. \par

	\end{flushleft}


\subsection{Section appendix}

	\begin{center}

\textbf{Interview Questionnaire}

	\end{center}

	\begin{flushleft}

\textbf{1. What is the Current System?} \par
- Done on Excel\par
- Shows product name and quantity in the front and back of vet\par
- when there are only 10 - 15 of the product left a new order is made\par
- when stock is low in the front, stock is moved from the back to the front\par

\textbf{2. What are the problems with the current system?}\par
- Data is sometimes not saved, which can cause problems\par
- Takes a long time to initially set up the system with all the products\par
- Every time a product is sold its data on the system has to be changed.\par
- There is a lot of data is needed to identify each product (Weight, 300g / 450g)\par
- Selling a product and changing the quantity left are two different processes.\par
- 

\textbf{3. What data is recorded in the system?}\par
- Product Name\par
- Quantity if sold in different quantities\par
- Weight if sold in different weights (i.e 500g / 750g)\par
- Quantity in the front of the vets(On display to the customer)\par
- Quantity in the back of the vets (Being stored in the storage room)\par

\textbf{4. How many clients buy products on a day to day basis?}\par
- Roughly about 10 - 20 people depend on how busy we are on that day.\par

\textbf{5. How many products do each client buy?}\par
- some customers may be advised to buy only one product, however some may be 5/6 depending on what problems their pet / pets have\par

\textbf{6. What features should the new system have?}\par
- The checkout system and the stock control system should be integrated together\par
- When a product is sold, its quantity is changed in the stock system automatically\par
- A search function to be able to search for a specific product.\par
- Categorize items so that alternative items can be given if there is a problem with the current product.\par
- A notification should be displayed if a product needs to be restocked\par
- The program should show which product and amount should be moved from the back of house to the front display.\par
- A picture of the product should be displayed so each product can be easily identified by a member of staff\par
- The product name, quantity, and weight (if needed) should be clearly displayed\par
- Because the checkout system and stock control should be integrated the price of each product should be displayed.\par
- The price of the product should be easily changed if it needs to be\par

\textbf{7. are any parts form the current system going to transfer to the new system?}\par
- The name of the product, quantity in the front of house, quantity in back of house will still be stored.\par
- When stock gets low in front of house, products will need to be moved from the storage room the the front.\par
- When total stock gets low more products will have to be ordered in.\par

\textbf{8. How often will data need to be input?}\par
- Only Once when the system is initially set up\par
- When a new product is going to be sold its data will be needed to be put into the system\par
- When the price of a product needs to be changed\par
- If a customer wants to buy more than one of a specific item the quantity will need to be entered\par

\textbf{9. what computer resources will the system have available to run on?}\par

\textbf{10. Is installing additional software?}\par
-Not an issue unless there is a problem installing the new software or  if i don't know how to install the new software.\par

	\end{flushleft}

\section{Investigation}


\subsection{The current system}

\subsubsection{Data sources and destinations}

\subsubsection{Algorithms}

	\begin{flushleft}
There are many algorithms used in selling products and controlling the stock of the product. \par

	\end{flushleft}

\begin{algorithm}[H]
\label{fig:repeat_pseudo_example}
	\caption{Adding Product to Total Price}
\begin{algorithmic}[1]
\SET{$Done$}{$False$}
\SET{$TotalPrice$}{$0$}
\While{$Not Done$}
\SEND{$"Please\ enter\ the\ price\ of \ the\ product (0.00)"$}
\RECEIVE{$Price$}
\SEND{$"Please\ enter\ the\ quantity\ of\ the\ product\ (0-100)"$}
\RECEIVE{$Quantity$}
\If{$Price == 0$}
\SET{$Done$}{$True$}
\SET{$TotalPricel$}{$TotalPrice + Price * Quantity$}
\EndIf
\EndWhile
\end{algorithmic}
\end{algorithm}


          \begin{flushleft}
Calculating Change:
	\end{flushleft}	
	
\begin{algorithm}[H]
	\caption{Calculating Change}
\begin{algorithmic}[1]
\SET{$Done$}{$False$}
\SEND{$''Please Enter Money Given"$}
\RECEIVE{$MoneyGiven$}
\SEND{$"Please enter the Price$}
\RECEIVE{$Price$}
\If {MoneyGiven < Price}
\SEND {''More Money Please''}
\ElsIf {MoneyGiven = Price}
\SEND{"Thank You!"}
\ElsIf {MoneyGiven\ > Price}
\SET {$Change$}{$MoneyGiven - Price$}
\SEND{$Change$}
\EndIf

\end{algorithmic}
\end{algorithm}																														
																										


\begin{algorithm}[H]
\label{fig:repeat_pseudo_example}
    \caption{Calculating Next Months Stock}
\begin{algorithmic}[1]
\SET{$PreviousSales$}{$SalesMadeLastMonth$}
\SET{$CurrentStock$}{$PreviousStock - PreviousSales$}
\SET{$CurrentSales$}{$SalesMadeThisMonth$}
\SET{$ProductName$}{$NameOfProduct$}
\SET{$PredictedSales$}{$CurrentSales + (CurrentSales - PreviousSales)$}
\If {$PredictedSales < 0$}
\SEND{$"No\ stock\ is\ required\ this\ month$}
\Else
\SEND{$ You\ should\ buy\ PredictedSales)\ of\ (ProductName)\ this\ month$}
\EndIf
\end{algorithmic}
\end{algorithm}
																							


\begin{algorithm}[H]
\label{fig:repeat_pseudo_example}
    \caption{Adding A New Product}
\begin{algorithmic}[1]
\SET{$ProductName$}{$NameOfProduct$}
\SET{$ProductID$}{$Product_List[ProductID]$}
\SEND{$"Does\ the\ product\ have\ a\ specific\ weight"$}
\RECEIVE{$Weight$}
\If{$Weight = "Yes"$}
\SEND{$Enter\ The\ Weight\ Of\ The\ Product$}
\RECEIVE{$WeightOfProduct$}
\SET{$ProductWeight$}{$WeightOfProduct$}
\EndIf
\SET{$ProductQuantity$}{$QuantityOfProductInStock$}
\SET{$ProductPrice$}{$PriceOfProduct$}
\SET{$ShopQuantity$}{$QuantityInTheShop$}
\SET{$StorageQuantity$}{$QuantityCurrentlyInStorage$}
\SEND{$"Is\ The\ Price\ of\ the\ product\ been\ reduced\ (On Offer)"$}
\RECEIVE{$Reduction$}
\If {$Reduction > 0$}
\SET{$ProductPrice$}{$ProductPrice * (Reduction / 100)$}
\EndIf 	
\end{algorithmic}
\end{algorithm}
																																			

%Categorizing A Product

\begin{algorithm}[H]
\label{fig:repeat_pseudo_example}
    \caption{Categorizing A Product}
\begin{algorithmic}[1]
\SEND{$" | DOG | CAT | BIRD | EQUINE | REPTILE | OTHER"$}
\SEND{$"Choose\ a\ Category\ in\ which\ the\ product\ comes\ under"$} 
\RECEIVE{$Category1$}
\If{$Category1 = "DOG"$}
	\SEND{$" | FOOD | HEALTH CARE | STESS AND ANXIETY"$}
	\SEND{$"Choose\ a\ Category\ in\ which\ the\ product\ comes\ under"$} 
	\RECEIVE{$Category2$}
	\ElsIf{$Category2 = FOOD$}
		\SEND{$" | WET FOOD | DRY FOOD |"$}
		\SEND{$Choose\ a\ Category\ in\ which\ the\ product\ comes\ under$}
		\RECEIVE{$Category3$}
		\SET{$ProductCategory$}{$CategoryList[Category1][Category2][Category3]$}
	\ElsIf {Category2 = HEALTH CARE}	
		\SEND{$" | ORAL | JOINTS | WORMING | EYE | SKIN | EAR |"$}
		\SEND{$Choose\ a\ Category\ in\ which\ the\ product\ comes\ under$}
		\RECEIVE{$Category3$}
		\SET{$ProductCategory$}{$CategoryList[Category1][Category2][Category3]$}
	\ElsIf{Category = STESS AND ANXIETY}
		\SEND{$" | DIFFUSER | TABLETS |  DROPS |"$}
		\SEND{$Choose\ a\ Category\ in\ which\ the\ product\ comes\ under$}
		\RECEIVE{$Category3$}
		\SET{$ProductCategory$}{$CategoryList[Category1][Category2][Category3]$}
\EndIf

\end{algorithmic}
\end{algorithm}


\begin{algorithm}[H]
	\caption{Moving Items From Storage to Front}
\begin{algorithmic}[1]
\SEND{$Please\ Enter\ Total\ Stock$}
\RECEIVE{$TotalStock$}
\SEND{$"Please enter the Wanted Amount in the Shop$}
\RECEIVE{$ShopAmount$}
\SET {MoneyGiven < Price}
\SEND {''More Money Please''}
\ElsIf {MoneyGiven = Price}
\SEND{"Thank You!"}
\ElsIf {MoneyGiven\ > Price}
\SET {$Change$}{$MoneyGiven - Price$}
\SEND{$Change$}
\EndIf

\end{algorithmic}
\end{algorithm}																														
																																																																												\subsubsection{Data flow diagram}

\subsubsection{Input Forms, Output Forms, Report Formats}

\subsection{The proposed system}

\subsubsection{Data sources and destinations}

\subsubsection{Data flow diagram}

\subsubsection{Data dictionary}

\subsubsection{Volumetrics}

\section{Objectives}

\subsection{General Objectives}

\subsection{Specific Objectives}

\subsection{Core Objectives}

\subsection{Other Objectives}

\section{ER Diagrams and Descriptions}

\subsection{ER Diagram}

\subsection{Entity Descriptions}

\section{Object Analysis}

\subsection{Object Listing}

\subsection{Relationship diagrams}

\subsection{Class definitions}

\section{Other Abstractions and Graphs}

\section{Constraints}

\subsection{Hardware}

\subsection{Software}

\subsection{Time}

\subsection{User Knowledge}

\subsection{Access restrictions}

\section{Limitations}

\subsection{Areas which will not be included in computerisation}

\subsection{Areas considered for future computerisation}

\section{Solutions}

\subsection{Alternative solutions}

\subsection{Justification of chosen solution}
