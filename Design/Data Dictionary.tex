\chapter{Design}

\section{Overall System Design}

\subsection{Short description of the main parts of the system}

\begin{itemize}
	\item Log-In Interface
	\begin{itemize}
		\item The Log In interface will be very simple and will ask for The Username and Password from the employee.
		\item There should be a button that allows the employee to reset their password if they forget it.
		\item None of the system should be accessible on or before the log in interface as it is a security measure
		\item The employee should be informed if they entered their username or password incorrectly
		\item When The User enters their password, it should be displayed as stars On the Log-In Interface
	\end{itemize}

	\item Product Search Interface
	\begin{itemize}
		\item There should be a search bar where the employee can enter the name of a product and there should be a table to display any matches with the information specified in the search bar.
		\item This search bar should be easily accessible by using a keyboard shortcut.
		\item When using the keyboard shortcut, the application should keep the user on their current page and the search bar should open to the side of the current page instead of taking them to a new page.
		\item Alternatively, there should be a main search page containing buttons with the different categories on so that the employee can find any products under one specific category.
		\item Clicking on a category button should redirect the employee to another selection of buttons named by the sub categories of that category.
		\item An image should be displayed aside from the product name so a product can be located even if the product name is not known.
		\item There should be a list down the left hand side of the Product search interface where the Categories will be listed so the employee can alternatively click on the categories which will expand with all the sub categories until a list of products under that sub category is found. This list is used so the employee doesn’t have to keep clicking back and forth between the category and sub category buttons.
		\item The Buttons should contain images, and should be a more simple way compared to using the list on the side of the page.
	\end{itemize}

	\item Creating a New Employee Account
	\begin{itemize}
		\item This feature should only be accessible on my clients (The administrators) log in.
		\item There should be fields to enter the Employee's first name, last name and email address. The password is not required at this stage as the employee will set it later.
		\item Once all the fields have been completed my client will be able to review the information entered and be able to edit it if necessary before finally creating the new account.
		\item Once the employee account has been created, the employee enters their login Name and enters the password ‘password’ at the Log In Interface. They will then be re-directed to a new page where they will have to enter and re-enter their new password (verification). 
	\end{itemize}
=======
\begin{itemize}
	\item Log-In Interface
	\begin{itemize}
		\item The Log In interface will be very simple and will ask for The Username and Password from the employee.
		\item There should be a button that allows the employee to reset their password if they forget it.
		\item None of the system should be accessible on or before the log in interface as it is a security measure
		\item The employee should be informed if they entered their username or password incorrectly
		\item When The User enters their password, it should be displayed as stars On the Log-In Interface
	\end{itemize}

\pagebreak

	\item Product Search Interface
	\begin{itemize}
		\item There should be a search bar where the employee can enter the name of a product and there should be a table to display any matches with the information specified in the search bar.
		\item This search bar should be easily accessible by using a keyboard shortcut.
		\item When using the keyboard shortcut, the application should keep the user on their current page and the search bar should open to the side of the current page instead of taking them to a new page.
		\item Alternatively, there should be a main search page containing buttons with the different categories on so that the employee can find any products under one specific category.
		\item Clicking on a category button should redirect the employee to another selection of buttons named by the sub categories of that category.
		\item An image should be displayed aside from the product name so a product can be located even if the product name is not known.
		\item There should be a list down the left hand side of the Product search interface where the Categories will be listed so the employee can alternatively click on the categories which will expand with all the sub categories until a list of products under that sub category is found. This list is used so the employee doesn’t have to keep clicking back and forth between the category and sub category buttons.
		\item The Buttons should contain images, and should be a more simple way compared to using the list on the side of the page.
	\end{itemize}

	\item Creating a New Employee Account
	\begin{itemize}
		\item This feature should only be accessible on my clients (The administrators) log in.
		\item There should be fields to enter the Employee's first name, last name and email address. The password is not required at this stage as the employee will set it later.
		\item Once all the fields have been completed my client will be able to review the information entered and be able to edit it if necessary before finally creating the new account.
		\item Once the employee account has been created, the employee enters their login Name and enters the password ‘password’ at the Log In Interface. They will then be re-directed to a new page where they will have to enter and re-enter their new password (verification). 
	\end{itemize}

\pagebreak
>>>>>>> branch 'master' of https://github.com/mattling9/COMP4Coursework.git

	\item Adding a New Member
	\begin{itemize}
		\item Similar to Creating a New Employee Account, adding a new member has fields where the information is entered.
		\item The new member can review the information before finalising it.
		\item The address and telephone number are also entered when adding a new member.
	\end{itemize}

	\item{Adding a New Product}
	\begin{itemize}
		\item There will be fields where the product name, size and price are entered.
		\item The category and sub category the product comes under will be selected in a dropdown menu for each category and sub category
		\item The Product ID is automatically assigned to each product so this does not have to be entered
		\item An image of the product is entered and the required file size and picture format is displayed.
		\item A message will be displayed if the image is larger than the maximum file size or if the image is the wrong format (i.e .jpg instead of .png)
		\item The resolution of the image will be changed when scaling the image so they’re all the same size. This means the employee doesn’t have to worry about making sure the dimension of every photo are the same.
	\end{itemize}



	\item Stock and Future Stock Prediction
	\begin{itemize}
		\item This page will be where the employee can view the current and predicted future stocks of each product.
		\item A pop up window will display when the stock of a product needs to be updated.
		\item Alternatively, the user can check which products need the stock updating on the reminders board.
		\item Here the user will be able to view the current stock of any particular product, the stock currently in each location, and the stock from the end of each month for the last 2 months.
	\end{itemize}
\end{itemize}

\subsection{System flowcharts showing an overview of the complete system}


\section{User Interface Designs}

\subsection{Hardware Specification}
	
\begin{flushleft}
The hardware specifications for my clients current computer are: \par

\begin{itemize}
\item Windows 7 Home Edition
\item Intel® Core™ i3-3240T Processor (2.9 GHz)
\item 4 GB DDR3 RAM
\item 1 TB HDD, 7200 rpm
\end{itemize}

I have decided that my client does not mandatorily need to purchase any futher hardware in order to run my new system. This is because my client's hardware is mor ethan capabale of running ym new system. Therefore, there shall be no cost for new hardware. My client's hardware is suitable for the purpose of my new system, the processor will be more than capable of running my system. However the computer is roughly 3 years old, meaning that errors within the registry may have occured which over time slows down the speed of the computer. This factor may or may not effect the speed in which my client can use my new system. My client has more than enough memory to run my system by itself. However, my client may or may not encounter small speed issues when using the system whislt other applications are running such as Google Chrome ect \ldots However most normal programs will not majorly effect the speed of the computer.  My client has a large 1 Terrorbyte Hard Disk Drive. The size of this drive is more than capable of storing all of the data within my system. My client does not have to make any mandatory hardware changes, however they may wish to make optional changes to improve the overall performance of the system. my client could purchase a new processor with equal or higher clock speeds, which may increase the overall speed of the computer depending on how old the computer is.  my client could also replace the 1TB Hard Disk Drive with a Solid State drive, this increases the Read / Write speeds of the data stored on the drive as the information is stored in flash memomry chips as apposed to magnetic coating on a platter. This means the information can be transfered much faster. However, solid state drives have an extremely high cost compared to Hard disk drives. These changes will be uneccessary and almost unoticable.

\section{Program Structure}


\subsection{Top-down design structure charts}

\subsection{Algorithms in pseudo-code for each data transformation process}

\subsection{Object Diagrams}

\subsection{Class Definitions}

\section{Prototyping}

\section{Definition of Data Requirements}

\subsection{Identification of all data input items}

\begin{itemize}
\item Product Name
\item Size
\item Category
\item Employee First Name
\item Employee Last Name
\item Employee Password
\item Employee Email
\item Date
\item Time
\item Title
\item Member First Name
\item Member Last Name
\item House No
\item Street
\item Town
\item City
\item County
\item Postcode
\item Telephone Number
\item Member Email
\item Location Name
\item Stock
\item Employee Username
\item Quantity
\end{itemize}

\subsection{Identification of all data output items}

\textbf{Output to user}
\begin{itemize}
\item Members First Name
\item Members Last Name
\item Members Email Address
\item Members Telephone Number
\item Employees First Name
\item Employees Last Name
\item Employees Email
\item Product Restock
\end{itemize}

\textbf{Output to Database}
\begin{itemize}
\item Product Name
\item Size
\item Category
\item Employee First Name
\item Employee Last Name
\item Employee Password
\item Employee Email
\item Date
\item Time
\item Title
\item Member First Name
\item Member Last Name
\item House No
\item Street
\item Town
\item City
\item County
\item Postcode
\item Telephone Number
\item Member Email
\item Location Name
\item Product Stock in each Location
\item Employee Username
\item Quantity
\end{itemize}
  

\end{itemize}

\subsection{Explanation of how data output items are generated}

    \begin{tabular}{|p{4cm}|p{4cm}|}
        \hline
	 \multicolumn{2}{|c|}{1NF} \\ \hline
	\textbf{Output} & \textbf{ How the Output is Generated}\\ \hline
	{Member First Name} & {The Data is displayed so the Member can review the information they have entered whilst creating a new Member}\\ \hline
	{Members Last Name} & {The Data is displayed so the Member can review the information they have entered whilst creating a new Member}\\ \hline
	{Members Email Address} & {The Data is displayed so the Member can review the information they have entered whilst creating a new Member}\\ \hline
	{Members Telephone Number} & {The Data is displayed so the Member can review the information they have entered whilst creating a new Member}\\ \hline
	{Employee First Name} & {The Data is displayed so the Employee can review the information they have entered whilst creating a new Employee}\\ \hline
	{Employee Last Name} & {The Data is displayed so the Employee can review the information they have entered whilst creating a new Employee}\\ \hline
	{Employee Password} & {The Data is displayed so the Employee can review the information they have entered whilst creating a new Employee}\\ \hline
	{Employee Email} & {The Data is displayed so the Employee can review the information they have entered whilst creating a new Employee}\\ \hline
	{Product Restock} & {If the total stock of a specific product has gotten lower than the minimum stock allowed.}\\ \hline
	{Product Name} & {Employee inputs Information}\\ \hline
	{Size} & {Employee inputs Information}\\ \hline
	{Category} & {Employee inputs Information}\\ \hline
	{Employee First Name} & {Employee inputs Information}\\ \hline
	{Employee Last Name} & {Employee inputs Information}\\ \hline
	{Employee Password} & {Employee inputs Information}\\ \hline
	{Employee Email} & {Employee inputs Information}\\ \hline
	{Date} & {Employee inputs Information}\\ \hline
	{Time} & {Employee inputs Information}\\ \hline
	{Title} & {Employee inputs Information}\\ \hline
	{Member First Name} & {Employee inputs Information}\\ \hline
	{Member Last Name} & {Employee inputs Information}\\ \hline
	{House No} & {Employee inputs Information}\\ \hline
	{Street} & {Employee inputs Information}\\ \hline
	{Town} & {Employee inputs Information}\\ \hline
	{City} & {Employee inputs Information}\\ \hline
	{County} & {Employee inputs Information}\\ \hline
	{Postcode} & {Employee inputs Information}\\ \hline
	{Telephone Number} & {Employee inputs Information}\\ \hline
	{Member Email} & {Employee inputs Information}\\ \hline
	{Location Name} & {Employee inputs Information}\\ \hline
	{Product Stock in each Location} & {Employee inputs Information}\\ \hline
	{Employee Username} & {Employee inputs Information}\\ \hline
	{Quantity} & {Employee inputs Information}\\ \hline
	{Total Price of Order} & {The sum of the prices of the products within the order}
   \end{tabular}


\subsection{Data Dictionary}

    \begin{tabular}{|p{3cm}|p{3cm}|p{2cm}|p{3cm}|p{3cm}|}
        \hline
        \textbf{Data} & \textbf{Data Type} & \textbf{Length} & \textbf{Validation} & \textbf{Example Data}\\ \hline
	ProductID & Integer & 100 & Range & 42 \\ \hline
	ProductName & String & 100 & length & OPTEX Ear Drops \\ \hline
	Size & Integer & 1000 & Range & 750 grams \\ \hline
          Price & Real & 250.00 & Range & 9.99 \\ \hline
          Catagory & String & 100 & Length & Dog food \\ \hline
          MemberID & Integer & 300 & Range & 24 \\ \hline
	Title & String & 5 & Length & Mrs. \\ \hline
          MemberFirstname & String & 20 & Length & Thomas \\ \hline
          MemberLastName & String & 20 & Length & Brennan \\ \hline
          House No. & Integer & 200 & Length & 66 \\ \hline
	Street & String & 50 & Length & Market Street \\ \hline
	Town & String & 50 & Length & Fordham \\ \hline
         City & String & 50 & Length & Ely \\ \hline
         County & String & 50 & Length & Cambridgeshire \\ \hline
         Postcode & String & 6 & Length & CB7 5DJ \\ \hline
         Telephone No. & Integer & 12 & Length & 07764563958 \\ \hline
	Member Email & String & 50 & Length & Market Street \\ \hline
	Employee First Name & String & 20 & Length & Matthew \\ \hline
	Employee Last Name & String & 20 & Length & Ling \\ \hline
	Employee Password & String & 18 & Length & Password123 \\ \hline
	Employee Email & String & 30 & Length & example@email.com \\ \hline
	Date & Integer & 10 & Range & 7/11/2014 \\ \hline
	Time & Integer & 5 & Range & 22:11 \\ \hline
	Location Name & String & 20 & Length & Storage Room \\ \hline
	
  \end{tabular}

\subsection{Identification of appropriate storage media}

\section{Database Design}

\subsection{Normalisation}

\subsubsection{ER Diagrams}

\subsubsection{Entity Descriptions}

\subsubsection{1NF to 3NF}

\pagebreak

\begin{center}
    \begin{tabular}{|p{4cm}|}
        \hline
        \textbf{Un-normalised (UNF)}\\ \hline
	{ProductID}\\ \hline
	{OrderID}\\ \hline
	{ProductName}\\ \hline
	{Size}\\ \hline
	{Catagory}\\ \hline
	{Price}\\ \hline
	{LocationID}\\ \hline
	{Location Name}\\ \hline
	{Stock}\\ \hline
	{Quantity}\\ \hline
	{Date}\\ \hline
	{Time}\\ \hline
	{MemberID}\\ \hline
	{Title}\\ \hline
	{First Name}\\ \hline
	{Last Name}\\ \hline
	{House / Flat No.}\\ \hline
	{Street}\\ \hline
	{Town}\\ \hline
	{City}\\ \hline
	{County}\\ \hline
	{Postcode}\\ \hline
	{Telephone Number}\\ \hline
	{Memebr Email}\\ \hline
	{Employee ID}\\ \hline
	{Employee FirstName}\\ \hline
	{Employee LastName}\\ \hline
	{Employee Password}\\ \hline
	{Employee Email}\\ \hline
   \end{tabular}
\end{center}	
	
	

\begin{center}
    \begin{tabular}{|p{4cm}|p{4cm}|}
        \hline
	 \multicolumn{2}{|c|}{1NF} \\ \hline
	\textbf{Repeating} & \textbf{ Non Repeating}\\ \hline
	\textbf{ProductID}  & \textbf{OrderID}\\ \hline
	\textbf{OrderID} & {Date}\\ \hline
	{Product Name} & {Time}\\ \hline
	{Size} & {MemberID}\\ \hline
	{Category} & {Title}\\ \hline
	{Price} & {Member First Name}\\ \hline
	{LocationID} & {Member Last Name}\\ \hline
	{LocationName} & {House / Flat No.}\\ \hline
	{Stock} & {Street}\\ \hline
	{Quantity} & {Town}\\ \hline
	{} & {City}\\ \hline
	{} & {County}\\ \hline
	{} & {Postcode}\\ \hline
	{} & {Telephone No.}\\ \hline
	{} & {Member Email}\\ \hline
	{} & {Employee ID}\\ \hline
	{} & {Employee First Name}\\ \hline
	{} & {Employee Last Name}\\ \hline
	{} & {Password}\\ \hline
	{} & {Employee Email}\\ \hline
   \end{tabular}
\end{center}

\pagebreak

\begin{center}
    \begin{tabular}{|p{4cm}|p{4cm}|p{4cm}|}
        \hline
	 \multicolumn{3}{|c|}{2NF} \\ \hline
	\textbf{ProductID}  & \textbf{OrderID} & \textbf{ProductID}\\ \hline
	\textbf{OrderID} & {Date} & {Product Name} \\ \hline
	{Quantity} & {Time} & {Size} \\ \hline
	{} & {MemberID} & {Category} \\ \hline
	{} & {Title} & {Price} \\ \hline
	{} & {First Name} & {Stock} \\ \hline
	{} & {Last Name} & {LocationID} \\ \hline
	{} & {House / Flat No.} & {Location Name} \\ \hline
	{} & {Street} & {Stock} \\ \hline
	{} & {Town} & {} \\ \hline
	{} & {City} & {} \\ \hline
	{} & {County} & {} \\ \hline
	{} & {Postcode} & {} \\ \hline
	{} & {Telephone Number} & {} \\ \hline
	{} & {Member Email} & {} \\ \hline
	{} & {EmployeeID} & {} \\ \hline
	{} & {Employee First Name} & {} \\ \hline
	{} & {Employee Last Name} & {} \\ \hline
	{} & {Password} & {} \\ \hline
	{} & {Employee Email} & {} \\ \hline	
    \end{tabular}
\end{center}

\begin{flushleft}
\begin{center}
    \begin{tabular}{|p{3cm}|p{3cm}|p{3cm}|p{2cm}|}
        \hline
	 \multicolumn{4}{|c|}{3NF} \\ \hline
	\textbf{Product ID}  & \textbf{Product-Order ID} & \textbf{Order ID} & \textbf{Employee ID} \\ \hline
	{Product Name} & \textit{ProductID} & \textit{Member ID} & {Employee First Name} \\ \hline
	{Size} & \textit{Order ID} & \textit{Employee ID} & {Employee Last Name} \\ \hline
	{Price} & {Quantity} & {Date} & {Password} \\ \hline
	{Stock} & {} & {Time} & {Employee Email} \\ \hline
	{} & {} & {} & {}\\ \hline
	{} & {} & {} & {}\\ \hline
	\textbf{Member ID} & \textbf{Product ID} & \textbf{Location ID} & \textbf{Category}\\ \hline
	{Title} & \textit{Location ID} & {Location Name} & {}\\ \hline
	{First Name} & {} & {} & {}\\ \hline
	{Last Name} & {} & {} & {}\\ \hline
	{House / Flat No.} & {} & {} & {}\\ \hline
	{Street} & {} & {} & {}\\ \hline
	{Town} & {} & {} & {}\\ \hline
	{City} & {} & {} & {}\\ \hline
	{County} & {} & {} & {}\\ \hline
	{Postcode} & {} & {} & {}\\ \hline
	{Telephone Number} & {} & {} & {}\\ \hline
	{Memebr Email} & {} & {} & {}\\ \hline
    \end{tabular}
\end{center}
\end{flushleft}


\section{Security and Integrity of the System and Data}

\subsection{Security and Integrity of Data}

\subsection{System Security}

\section{Validation}

\section{Testing}

\begin{landscape}
\subsection{Outline Plan}

\begin{center}
    \begin{tabular}{|p{2cm}|p{5cm}|p{5cm}|p{4cm}|}
        \hline
        \textbf{Test Series} & \textbf{Purpose of Test Series} & \textbf{Testing Strategy} & \textbf{Strategy Rationale}\\ \hline
        Example & Example & Example & Example \\ \hline
    \end{tabular}
\end{center}

\subsection{Detailed Plan}

\begin{center}
    \begin{longtable}{|p{1.5cm}|p{2.5cm}|p{2.5cm}|p{2cm}|p{2cm}|p{2cm}|p{2cm}|p{2cm}|}
        \hline
        \textbf{Test Series} & \textbf{Purpose of Test} & \textbf{Test Description} & \textbf{Test Data} & \textbf{Test Data Type (Normal/ Erroneous/ Boundary)} & \textbf{Expected Result} & \textbf{Actual Result} & \textbf{Evidence}\\ \hline
        Example & Example & Example & Example & Example & Example & Example & Example \\ \hline
    \end{longtable}
\end{center}
\end{landscape}
